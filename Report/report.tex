\documentclass[a4paper, 12pt]{report}
\usepackage[left=2.5cm, right=2.5cm, top=3cm, bottom=3cm]{geometry}

\usepackage{xcolor}
\usepackage{amsmath, amssymb, amsfonts, amsthm}
\usepackage{url}
\usepackage{graphicx}

% use spanish template
% \usepackage[spanish]{babel}

% add bibliographics quotes
\usepackage{cite}

% insert source code in latex
\usepackage{listings}
\usepackage{color}

\definecolor{gray97}{gray}{.97}
\definecolor{gray75}{gray}{.75}
\definecolor{gray54}{gray}{.45}
\lstset{
  frame=Ltb,
  framerule=0pt,
  aboveskip=0.5cm,
  framextopmargin=3pt,
  framexleftmargin=0.4cm,
  framesep=0pt,
  rulesep=.4pt,
  backgroundcolor=\color{gray97},
  rulesepcolor=\color{black},
  %
  showstringspaces=true, columns=fullflexible, basicstyle=\ttfamily,
  stringstyle=\color{orange}, commentstyle=\color{gray45},
  keywordstyle=\bfseries\color{green!40!black},
  %
  numbers = left, numbersep=15pt, numberstyle=\tiny, numberfirstline=false,
  breaklines=true,
}

\lstnewenvironment{listing}[1][]
{\lstset{#1}\pagebreak[0]}{\pagebreak[0]}

\begin{document}
\title{\bf HULK Interpreter documentation}
\author{Eduardo Brito Labrada}
\date{\today}
\maketitle

\begin{abstract}
  In computer science, an {\bf interpreter} is a computer program that directly executes
  intructions written in a programming or scripting language, without requiring them previously
  to have been compiled into a machine language program. An interpreter generally uses one of the
  following strategies for program execution:

  \begin{enumerate}
    \item Parse the source code and perform behavior directly;
    \item Translate source code into some efficient intermediate representation or object code
          and inmediately execute that;
    \item Explicitly execute stored precompiled bytecode made by a compiler and matched with the
          interpreter Virtual Machine.
  \end{enumerate}

  In this project we will focus on the first strategies of those to create a interpreter for
    {\em Havana University Language for Kompilers (HULK)}. First we will define the basic syntax of the
  language and then we will show how the interpreter works in its entirety.
\end{abstract}

\tableofcontents

\newpage

\section*{Introduction}
\addcontentsline{toc}{section}{\bf Introduction}

HULK is a didactic, type-safe, object-oriented and incremental programming language. This is a simplified 
version of HULK where we will be implementing a subset of this programming language. In particular, this 
subset consists only of expressions that can be written on one line. 

\subsection*{Expressions}
\addcontentsline{toc}{subsection}{Expressions}

HULK is a ultimately an expression-based language. Most of the syntactic constructions in HULK are expressions,
including the body of all functions, loops and other block of code.

The body of a program in HULK always end with a single global expression (and, if necessary, a final semicolon\footnote{In this 
version of HULK all expressions end with a single semicolon}) that serves as the entrypoint of the program.

\subsubsection*{Arithmetic expressions}
\addcontentsline{toc}{subsubsection}{Arithmetic expressions}

HULK defines three types of literal values: {\bf numbers}, {\bf strings} and {\bf booleans}. Numbers are $32$-bit floating-point
and support all basic arithmetic operations with the usual semantic: {\tt +} (addition), {\tt -} (subtraction), {\tt *} (multiplication),
{\tt \slash} (floating-point division), {\tt \^{}} (power), and parenthesized sub-expressions. 

\subsubsection*{Strings}
\addcontentsline{toc}{subsubsection}{Strings}

Strings literals in HULK are defined within enclosed double-quotes ({\tt "}). A double-quote
can be included literally by escaping it ($\backslash${\tt "}), and other escaped characters are $\backslash${\tt n} for line endings and $\backslash${\tt t} for tabs.

\end{document}